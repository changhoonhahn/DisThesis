%Through the connection between galaxies and their host dark matter structures, 
Through their connection with dark matter structures, galaxies act as 
tracers of the underlying matter distribution in the Universe. Their 
%observed spatial distribution allows us to make precise measurements of large scale 
observed spatial distribution allows us to precisely measure large scale 
structure and effectively test cosmological models that explain the content, 
geometry, and history of the Universe. Current observations from galaxy 
surveys such as the Baryon Oscillation Spectroscopic Survey
have already probed vast cosmic volumes with millions of galaxies 
and ushered in an era of precision cosmology. The next surveys will 
probe over an order of magnitude more. With this unprecedented statistical 
power, the bottleneck of scientific discovery is in the methodology.
% need something that sounds better than this sentence.
In this dissertation, I address major methodological challenges in 
galaxy clustering analyses that prevent us from realizing .
Observational and instrumentational constraints restrict the 
constraining power and bias galaxy clustering analyses. I develop
data-driven and analytic methods for robustly accounting for these
systematic effects in galaxy clustering analyses. 
Standard probabilistic inference carry untested and incorrect assumptions. 
I apply innovative inference to galaxy clustering analyses. 
% galaxy -halo connection.




present galaxy 
clustering analyses with incomplete data. Beyond restricting the 
constraining power, if not properly accounted for, these systematic 
effects significantly bias galaxy clustering measurements. I develop 
data-driven and analytic methods for robustly accounting for these
systematic effects in galaxy clustering analyses. 


used to constrain cosmological parameters 
and investigate the connection between galaxies and their host dark 
matter structures.

%that prevent us from realizing the full potential of observations for constraining cosmology. 



I will present how major methodological challenges can be solved with robust 
treatment of systematics, higher order statistics, innovative approaches to 
probabilistic inference, and improved understanding of the galax-halo connection. 
By overcoming these challenges and unlocking the full potential the next galaxy 
surveys, I will present how we can measure the growth of structure and total 
neutrino mass with unprecedented precision.


%The study of exoplanets has been revolutionized in recent years thanks, in
%large part, to new data collected by NASA's \emph{Kepler} Mission.
%The Mission has enabled the discovery of thousands of planets orbiting stars
%throughout the Galaxy.
%These discoveries span orders of magnitude in physical parameter space but
%many of the most physically interesting questions remain open.
%The deepest of these questions is: how common are planetary systems like our
%own Solar System?
%In this dissertation, I approach this question from several different angles
%and make inferences about the frequency and distribution of planets based on
%the large, publicly-available datasets from the \emph{Kepler} and \emph{K2}
%Missions.
%
%I develop two powerful and practical methods for mining for planetary transit
%signals in the hundreds of thousands of stellar light curves measured by
%\emph{Kepler}.
%The first method is designed to find planets using the data from the
%\emph{K2} phase of the Mission where systematics introduced by the instrument
%dominate the measurements.
%Applying this method to the first publicly available dataset from \emph{K2}, I
%published more than thirty new exoplanet candidates.
%The second transit search technique is designed to find transits of planets
%with orbital periods longer than the four year baseline of the \emph{Kepler}
%Mission.
%These are interesting planets because they are expected to have the largest
%dynamical influence on the formation and evolution of their planetary systems
%but, to date, no systematic search for these signals has been published.
%I demonstrate that this method is robust and tractable and make predictions
%for the planet yields in the \emph{Kepler} dataset.
%
%I derive a general framework for making justified probabilistic inferences
%about the population of planets based on noisy and incomplete catalogs of
%exoplanet measurements.
%Applying this to a previously published catalog of exoplanets orbiting stars
%like our Sun, I measure the joint period--radius distribution of these
%planets taking into account survey selection effects and the large
%measurement uncertainties.
%Despite the fact that this catalog includes no true Earth analogs, I use the
%detected systems and weak smoothness assumptions about the underlying
%distribution to make a probabilistic estimate of the frequency of Earth-like
%planets.
%
%The main contributions of this dissertation are the development of methods for
%probabilistic and the release of open source implementations.
%One of these methods is \emph{emcee}, a method for Markov Chain Monte Carlo
%(MCMC) sampling of probability distributions.
%MCMC has been a popular method for approximate inference in astronomy for well
%over a decade but most implementations require extensive hand tuning in order
%to achieve acceptable performance on all but the simplest problems.
%Thanks to its affine-invariant sampling algorithm the \emph{emcee} method
%performs efficiently for many real problems in astronomy.
%The code has an active user base and online community of contributors.
