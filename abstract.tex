%Through the connection between galaxies and their host dark matter structures, 
Through their connection with dark matter structures, galaxies act as 
tracers of the underlying matter distribution in the Universe. Their 
%observed spatial distribution allows us to make precise measurements of large scale 
observed spatial distribution allows us to precisely measure large scale 
structure and effectively test cosmological models that explain the content, 
geometry, and history of the Universe. Current observations from galaxy 
surveys such as the Baryon Oscillation Spectroscopic Survey
have already probed vast cosmic volumes with millions of galaxies 
and ushered in an era of precision cosmology. The next surveys will 
probe volumes over an order of magnitude larger. With this unprecedented 
statistical power, the bottleneck of scientific discovery is in the methodology.
% need something that sounds better than this sentence.

In this dissertation, I address major methodological challenges in
constraining cosmology with the large-scale spatial distribution of 
galaxies. I develop a robust framework for treating systematic effects, 
which significantly bias galaxy clustering measurements. I apply new 
innovative approaches to probabilistic parameter inference that 
challenge and test incorrect assumptions of the standard approach. 
Furthermore, I use precise predictions of structure formation from
cosmology and observations of galaxies during the last eight billion 
years to develop detailed models of how galaxies are impacted by 
their host dark matter structures. These models provide key 
insight into the galaxy-halo connection, which bridges the gap
between cosmology theory and observations. They also answer crucial 
questions of how galaxies form and evolve. 
The developments in this dissertation will help unlock the full 
potential of future observations and allow us to precisely test 
cosmological models, General Relativity and modified gravity
scenarios, and even particle physics theory beyond the Standard 
Model.  
