\chapter*{Conclusion}\addcontentsline{toc}{chapter}{Conclusion}

Insert conclusion here. 


%In this dissertation, we study the population of exoplanets using data from
%NASA's \kepler\ Mission and the re-purposed \KT\ Mission.
%We develop and apply novel techniques to discover previously unknown planets
%and planet candidates (\chapname s~\chapalt{ketu} and~\chapalt{peerless}).
%We present a robust probabilistic framework for making inferences about the
%population of exoplanets based on the noisy and incomplete catalogs derived
%from transit surveys (\chap{exopop}).
%The main contributions of this dissertation are methodological and each
%\chapname\ is accompanied by open source software implementing the methods.
%
%In the spirit of tool development and open source software, \Chap{emcee} is
%describes \project{emcee}, a general purpose Markov Chain Monte Carlo sampler
%that, since its release \citep{Foreman-Mackey:2013}, has become one of the
%most popular tools for probabilistic inference in astronomy.
%This method was originally proposed by \citet{Goodman:2010} and it was
%designed to sample problems efficiently with little tuning even when the
%parameter space is poorly conditioned.
%This feature is especially useful for problems in astronomy where the physical
%parameters often vary (and covary) over many orders of magnitude.
%The \project{emcee} implementation offers a small performance gain by deriving
%a parallelizable version of the original algorithm and a user-friendly and
%well documented Python interface.
%In practice, this method doesn't scale well to large numbers of dimensions
%($\gtrsim 50$) but it has been shown to work out-of-the-box on a large class
%of typical astronomy problems.
%
%In \Chap{exopop}, we derive a hierarchical method for inferring the population
%of exoplanets based on a catalog of planets with a non-trivial completeness
%function and large measurement uncertainties.
%This method builds on the importance sampling technique originally derived by
%\citet{Hogg:2010a} to make a clean histogram from noisy measurements.
%Applying this population inference method to a catalog of planet candidates
%transiting Sun-like stars \citep{Petigura:2013}, we make a prediction for the
%rate of Earth analogs.
%This prediction is substantially lower than earlier predictions based on the
%same catalog.
%We demonstrate that this discrepancy is caused by both the treatment of the
%observational uncertainties and the choice of extrapolation function.
%
%In Summer 2014, the \kepler\ spacecraft was re-purposed and it began taking
%data for the \KT\ Mission.
%The pointing accuracy in this mode is substantially degraded relative to the
%original Mission but, in \chap{ketu}, we demonstrate that these light curves
%can still be used to systematically search for transiting exoplanets.
%By building a flexible data-driven model for the systematic variability in the
%light curves of the stars and combining this with an approximate linear
%transit model, we derive a transit search algorithm where the systematics
%model is marginalized for every hypothesis.
%This enables the discovery of transit signals with amplitudes smaller than the
%pointing-induced variability.
%In \chap{ketu}, we announce the discovery of 36 planet candidates transiting
%33 stars.
%Of these candidates, 18 have been validated as bona fide planets and 6 have
%been identified as likely astrophysical false positives
%\citep{Crossfield:2015, Montet:2015, Armstrong:2015a}.
%
%Finally, in \chap{peerless}, we present a novel method for detecting the
%transits of planets with orbital periods longer than the baseline of
%observations.
%Existing transit search methods are blind to these long periods because it is
%technically difficult to distinguish a single transit from coincidental
%variability in light curves.
%This constraint is not acceptable for forthcoming surveys like \KT\ and \tess\
%where the observation baselines are shorter than the periods of the most
%important planets for studies of dynamics and habitability.
%We apply a supervised classification algorithm, implemented using a set of
%Random Forest classifiers trained on simulated transits, to predict the
%``class'' (\texttt{transit} or \texttt{no transit}) of every section of light
%curve.
%Using this method, we announce the discovery of a convincing single transit
%candidate with a radius of $\sim 2\,R_\mathrm{J}$.
%
%The ultimate goal of this research program is an improved understanding of the
%population of exoplanets at the currently uncharted extremes of parameter
%space, especially pushing to long periods.
%This dissertation represents a step in this direction but there are some
%conspicuously missing pieces in the methods presented in these pages.
%One major shortcoming is that neither \chap{ketu} or \chap{peerless} realized
%the dream of a fully automated search.
%In both projects, a final stage of manual vetting was required to reach the
%target precision.
%This is unacceptable if we want to make rigorous inferences of the population
%of planets because human components of a pipeline can't be stress-tested and
%characterized for consistency and performance.
%The main barrier to completely automated search is that we don't have an
%acceptable generative model for the signals that are mis-classified by the
%search algorithms and we can never be completely sure that any section of
%light curve \emph{does not have any transits}.
%This goal of fully automated transit discovery will become even more important
%as new datasets continue to roll in from \KT, \tess, and \plato.
%This should be a focus of large scale transit programs over the next years.

