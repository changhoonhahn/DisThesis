\chapter*{Conclusion}\addcontentsline{toc}{chapter}{Conclusion}
%
%\todo{
%something that references the intro and answers it  \\
%In this dissertation, we have tackled key methodological challenges 
%in galaxy clustering analyses and galaxy evolution with robust treatment of systmeatics, innovative
%approaches to inference, and improved models of the galaxy-halo 
%connection.
%} 
%
%
%\todo{Summary of \Chap{fc}}
%
%\todo{Summary of \Chap{abc}}
%
%\todo{Summary of \Chap{galenv}} % galaxy-halo connection from an observational view
%
%\todo{Summary of \Chap{galhalo}} % galaxy-haloc onnection from a data-driven LCDM view

Over the next decade future surveys, namely eBOSS and DESI, will expand the 
cosmic volumes probed with redshifts by an order of magnitude. They have the 
potential to measure the growth of structure and constrain cosmological 
parameters with unprecedented precision. The main challenges for realizing 
their full statistical power are methodological. The frameworks I present
in this dissertation -- robust treatment of systematics, innovative approaches 
to accurate inference, and improved models of the galaxy-halo connection -- can be 
extended to these future surveys and used to tackle key methodological 
challenges. 
%by robust treatment of systematics, accurate modeling, and higher order statistics.
%eBOSS and DESI will expand the cosmic volume probed with redshifts by an order of magnitude. 
%They have the potential to measure the growth of structure and \mneut~with unprecedented precision.
%eBOSS and DESI will probe unprecedented cosmic volumes with galaxy redshifts and have the potential to measure the growth of structure and \mneut~with extraordinary precision. 
%Galaxy clustering and thereby the growth of structure from RSD and \mneut ~can be measured with unprecedented precision. 
%The main obstacles for realizing the full statistical power are methodological and can be solved by robust treatment of systematics, accurate probabilistic inference, and higher order statistics.\\ \vspace{-3mm}

For instance, observational systematics such as fiber collisions will continue 
to impact galaxy clusteing analyses of eBOSS and DESI, which will utilize 
fiber-fed spectrographs. As described in~\Chap{fc}, due to the impact that fiber 
collisions have on small scales, much of the statistical gains from eBOSS and 
DESI will be {\em wasted} if they are not properly account for in analyses. 
In fact, in eBOSS and DESI the systematics will be more complicated with multiple 
classes of target objects and automated fiber positioning 
\citep{Cahn:2015_desifib, Dawson:2015aa}. But the methods from ~\Chap{fc} 
can be extended to both surveys.

Furthermore, in~\Chap{abc}, we revealed deviations between the ABC posterior 
probability distribution and the standard Gaussian pseudo-likelihood approach to 
inference -- even in the narrower context of halo occupation modeling. Yet 
there have not been direct investigations on the impact of the standard 
assumptions on more general cosmological parameter constraints. With the 
increased statistical power of future surveys, quantifying the impact of these 
assumptions in our inference is critical for unbiased constraints. While
tractability of forward modeling the data has been an obstacle for adopting 
ABC, new models aimed at the next galaxy surveys, are making promising 
strides. 

Finally, as we describe in~\Chap{galenv}, observations of galaxies have 
firmly established a global view of galaxy properties out to $z{\sim}1$. 
As in \Chap{galhalo}, precise predictions of hierarchical growth of 
structure from $\Lambda$CDM can be used to constrain key elements of 
galaxy evolution in a data-driven and statistical fashion. The introduction 
of Integral Field Unit observations (\emph{e.g.} MaNGA) and larger 
galaxy samples (\emph{e.g.} DESI Bright Galaxy Survey) offer exciting 
opportunities to extend the works of \chapname s~\chapalt{galenv} and~\chapalt{galhalo}
and construct better models of the galaxy-halo connection.

Each aspect of my dissertation will be instrumental for exploiting the full 
potential of future surveys and making more precise measurements of the growth
rate of structure, the cosmological parameters, and thus tests of General Relativity 
and modified gravity scenarios. 
%Measuring this redshift-space distortion (RSD) allows us to infer the growth of structure, which we can then use to test GR and modified gravity scenarios. 
%doubly instrumental for extracting accurate and precise \mneut ~constraints from eBOSS and DESI.
Furthermore, galaxy clustering also provides a unique window to probe 
fundamental physics --- {\em i.e.} the total neutrino mass (\mneut).
Extending the methods from my dissertation to future surveys will allow 
us to better measure the imprints of neutrinos on LSS and produce 
tigher constraints on \mneut. A tighter 
upper limit on \mneut ~is essential to distinguish between the neutrino mass 
hierarchies and will provide an important input for particle physics theory 
beyond the Standard Model.
