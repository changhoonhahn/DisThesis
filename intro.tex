\chapter*{Introduction}\addcontentsline{toc}{chapter}{Introduction}
Amidst the countless stars and galaxies we observe in the Universe lie 
undetected structures of dark matter, orders of magnitude larger than 
the luminous objects they engulf. These vast invisible structures began 
in the very early Universe, as quantum fluctuations in the aftermath of 
the Big Bang. 
%These vast invisible structures began as quantum fluctuations in the beginning of the Universe in the aftermath of the Big Bang. 
During the subsequent period of inflation (\todo{cite?}), these primordial fluctations 
were amplified by the accelerated expansion of the Universe and then 
propagated through gravitational instability for billions of years. 

Despite constituting most of the matter in the Universe, dark matter 
has yet to be directly observed. In fact, it can only be studied through 
its gravitational interactions with luminous baryons — the matter that 
constitutes stars, galaxies, and celestial objects that emit light. 
In a way, the galaxies we observe in the cosmic volumes probed by our 
telescopes act as illuminated beacons tracing the vast dark matter 
terrains of the Universe.

Over the past decade, spectroscopic redshift surveys like the 
Baryon Oscillation Spectroscopic Survey (BOSS\todo{cite}) have 
exploited these galactic beacons to map out the cosmic structure
of the Universe. Precise measurements of distance and growth 
of large-scale structure (LSS) from these surveys, provide tests 
of cosmological models that desribe the content, geometry and history 
of the Universe. \todo{By analyzing measurements of LSS,}


Through these galaxies, we can explore the properties of the underlying dark matter
and the growth of their structure. Measurements that quantify these properties allow us to make
precise calculations of cosmological parameters, which quantify the content, geometry, and
expansion history of the Universe. Ultimately the constraints we measure on these parameters,
enlighten us on the properties of dark energy, which remains one of the most crucial unsolved
questions in cosmology. Certainly these precise cosmological measurements require a profound
understanding of the formation and evolution of galaxies. Unfortunately there is no clear
narrative of galaxy formation and evolution due to the complex, non-linear, and stochastic nature
of the physical processes that govern them.

In fact, galaxy formation and evolution remain another central unsolved questions in
astrophysics and cosmology. However, since galaxies are enveloped in the massive gravitational
wells of their host dark matter structures, the underlying dark matter of galaxies undoubtedly
plays a crucial role in their formation and evolution. Therefore, with its implications on the most
crucial questions in both cosmology and our understanding of galaxies, the interactions between
galaxies and their host dark matter environments pose some of the most impactful questions,
questions that I seek to answer in my dissertation. 


%\section{$\Lambda$CDM cosmology}
\section{Large Scale Structure} \label{sec:lss}
From the early Universe, primordial quantum fluctuations grow into the 
large-scale structures of the Universe we observe today through the different
epochs of cosmic history and by gravitational instability. In this section, I briefly 
describe the simplified ({\em linear}) theory of this evolution and explain core 
concepts of LSS cosmology using galaxies. Lets begin by defining the matter 
overdensity field (or density fluctuation) at comoving position ${\bf r}$: 
\beq \label{eq:delta}
\delta({\bf r}) = \frac{\rho({\bf r}) - \bar{\rho}}{\bar{\rho}}, 
\eeq
where $\rho({\bf r})$ and $\bar{\rho}$ are the density field and mean 
density respectively. In Fourier space ($k$-space), Eq.~\ref{eq:delta} can 
be Fourier transformed, 
\beq
\delta({\bf k}) = \int \frac{{\rm d^3}{\bf r}}{(2\pi)^3} e^{-i{\bf k}\dot{\bf r}}\;\delta({\bf r}).
\eeq
For describing the evolution of the overdensity field, Fourier space is generally 
favored over configuration space because, as derived later in the section,
the Fourier modes of $\delta$ evolve independently in linear theory -- 
\emph{i.e.} on large scales.

The information in the overdensity field is often quantified using its 
$N$-point statistics (\todo{cite Peebles bernardeua}). In fact, the two-point
statistic is one of commonly used tool in large scale structure studies.
This two-point statistic (also referred to as the correlation function) is
defined as, 
\beq
\xi(r) = <\delta({\bf x})\delta({\bf x} +  {\bf r})>
\eeq
and in Fourier space as,
\beq
<\delta({\bf k})\delta({\bf k'})> = (2\pi)^3 P(k) \delta^{D}({\bf k}+{\bf k'}).
\eeq
$\delta^{D}$ is the Dirac delta function and $P(k)$ is the {\em powerspectrum}, 
the Fourier transform of $\xi$. In principle, $\xi$ and $P(k)$ contain the same 
information. In practice, however, analyzing $\xi$ versus $P(k)$ carry different 
caveats (\todo{cite FKP}). Throughout this section, and also the dissertation, 
I will mainly focus on the powerspectrum. 

The evolution of the dark matter overdensity field can be derived (on 
sub-horizon scales) as follows. For pressureless dark matter, the equation of motion
can be derived from the continuity, Euler, and Poisson equations
\beqa 
\frac{\partial \rho}{\partial t} + \nabla \dot \rho {\bf u} = 0  \\ 
\frac{\partial {\bf u}}{\partial t} + ({\bf u} \dot \nabla) \dot {\bf u} - \nabla\Phi = 0 \\ 
\nabla^2\Phi - 4 \pi G \rho = 0 
\eeqa
to 
\beq \label{eq:meszaros}
\frac{\partial^2 \delta}{\partial t^2} + 2 \frac{\dot{a}}{a} \frac{\partial \delta}{\partial t} - 4 \pi G \bar{\rho} \delta = 0.
\eeq
${\bf u}$ is the velocity field, $\Phi$ is the gravitational potential, and $a$ 
is the scale factor. For a detailed derivation I refer readers to \todo{cite Peebles 
or Dodelson}. Eq.~\ref{eq:meszaros} a second order differential equation, therefore,
the solution can be written as 
\beq 
\delta({\bf r}, t) = D^{(+)}(t) A({\bf r}) + D^{(-)}(t) B({\bf r}).
\eeq
The density flucation has two components: a growing mode $D^{(+)}$ and a decaying 
mode $D^{(-)}$. The decaying mode decreases with time and its contribution becomes negligible
leaving only the growing mode in the late Universe. Now in order to quantify
the evolution of the growing mode $D^{(+)}$, one commonly used quantity is the 
``growth rate of structure'': 
\beq \label{eq:f_growth}
f = \frac{ d {\rm ln}\;D^{(+)}}{d {\rm ln}\;a}. 
\eeq
This growth rate of structure is a key quantity in LSS cosmology for testing different 
cosmological models and theories of gravity. $f$ will be discussed further in Section~\ref{sec:rsd}.

From the early Universe the density fluctuations evolve through different epochs in 
cosmic history: inflation, radiation-dominated, and matter-dominated eras. Each of 
these periods leave an imprint on the evolution of $\delta$. Fig.~\ref{fig:lifo}, 
marks the different eras in the early Universe and plots how the physical scale of 
the Universe, represented by the Hubble radius, evolves with $a$. 

During inflation, the Hubble radius remains constant. Afterwards the Universe 
becomes radiation dominated. Based on the Friedmann equations the Hubble radius 
during the radiation dominated era is approximately $\propto a^{2}$. The Universe then 
becomes matter dominated and the Hubble radius is approximately $\propto a^{3/2}$. 
Meanwhile, the physical scale of perturbations is 
$\lambda_{phys} = \lambda_{comov}\ a(t) \propto a(t)$. As Fig.~\ref{fig:lifo} 
illustrates, perturbations exit the Hubble radius during inflation then 
reenter the Hubble radius later on. Depending on the physical scale of the 
perturbation, it can enter either during the radiation dominated  
(smaller scale) or matter dominated (larger scale) era. 


The physical scale of perturbations that enter the horizon at the time of 
matter-radiation equality ($a(t) = a_{eq}$), $\lambda_{eq}$ is $\sim 500 h^{-1}Mpc$.
The perturbations that enter before the matter-radiation equality during 
the radiation dominated era, have physical scales $\lambda_{phys} < \lambda_{eq}$. 
These smaller scale perturbations, todo{explain the supression} 
On the other side, the larger scale perturbations with $\lambda_{phys} > \lambda_{eq}$
enter after matter-radiation equality during the matter dominated epoch. These 
perturbations do {\em not} experience the suppression of growth of the radiation 
dominated era. Therefore, as the overdensity field goes through these epochs, its 
growth on small scale is suppressed by a factor of $\sim k^4$.  


In practice, this scale dependent evolution of the density fluctuation is 
quantified through the ``transfer function'' $T(k)$ (\todo{cite all the transfer function papers}). After inflation the 
powerspectrum of the density fluctation can be summarized by: 
\beq
P_{inf}(k) \propto k^{n_s}
\eeq
where $n_s \sim 1$ \todo{cite inflation papers: Harrison (1970), Zel’dovich (1972) and Peebles Yu (1970) Komatsu et al. 2011}. Then powerspectrum of the density 
fluctuation in the late Universe can be expressed as 
\beq
P(k) \propto k^{n_s} T^2(k) D^2(k). 
\eeq
$D(k) \equiv D^(+)$ from earlier this section. 

\begin{figure*}
\begin{center}
\includegraphics[width=\textwidth]{figs/lifo.png}
\caption{Schematic diagram that illustrates the evolution of the Hubble radius
(Above) The evolution of the Hubble radius (solid line) during inflation (flat), radiation
domination, and matter domination (note inflection). Dashed, dotted, and dot-dashed
lines show the physical length of three constant comoving scales. The scale corresponding
to the current Hubble radius cH−1
0 first “left the horizon” about 60 e-folds before the end
of inflation (open circle).
} \label{fig:lifo}
\end{center}
\end{figure*}

\todo{paragraph about how because D and T depend on cosmological parameters 
    so powerspectrum measurements of the matter density fluctuation can be 
    used a tests of the cosmological models.
}
So powerspectrum measurements of the matter density fluctuations can be compared
to predictions of various cosmological models in order to produce constraints 
on cosmological parameters, better understand dark energy, and test theories of 
gravity. Unfortunately, most of the matter in the Universe is in the form of dark 
matter and does not interact with radiation. 

Observers cannot measure the spatial statistics (or clustering) of dark matter 
directly. Instead, we measure the clustering of galaxies and quasars, which trace
the underlying matter distribution. The smoothed galaxy/quasar density field is 
approximated by a local function of the matter density field
\beq
\delta_g({\bf r}) = F( \delta({\bf r} ). 
\eeq
This function can then be expanded Taylor series,
\beq
\delta_g({\bf r}) = \sum\limits_{k=0}^{\infty} \frac{b_k}{k!} \delta^k. 
\eeq
$b_1$ is referred to as the linear bias factor and $b_0$ is chosen so that
$<\delta_g> = 0$. To linear order, 
\beq
P_g(k) = b_1^2 P(k). 
\eeq
The primary subpopulation of galaxies used so far for LSS studies are luminous 
red galaxies (\todo{Eistenstien apper}). These galaxies have $b_1 > 1$, which 
means they are {\em biased} tracers of the matter distribution (\todo{cite Zehavi 2005, Sheldon 2009, Gastanage 2009}). 
This bias is caused by the fact that luminous galaxies reside in larger potential 
wells, which have stronger clustering properties than than less massive ones (\todo{manera 2010}). 

Based on the simplified derivation of this section, once we have the spatial 
distribution of galaxies or quasars we can derive the clustering of the matter
distribution and then infer cosmological constraints. In practice, however, a
number of factors complicate this procedure. One key complication is redshift-space
distortions, which will be discussed in the following section.

\section{Redshift-Space Distortions} \label{sec:rsd}
Spectroscopic redshifts surveys, such as 2dFGRS, SDSS, and BOSS, have mapped out
millions of distant galaxies. Current surveys such as Extended Baryon Oscillation 
Spectroscopic Survey (eBOSS; \citealt{Dawson:2015aa}), and future surveys such as 
the Dark Energy Survey Instrument (DESI; \citealt{Schlegel:2011aa, Morales:2012aa, Makarem:2014aa}) 
and the Subaru Prime Focus Spectrograph (PFS; \citealt{Takada:2014aa}), will 
continue to map out millions more. These surveys dominate LSS studies and have/will 
been critical for inferring precise cosmological constraints. As their name suggest, 
however, these {\em redshift} surveys do not directly measure the position of 
galaxies, but rather the angular positions (right ascension and declination) and
redshift of galaxies. 

Redshifts from spectroscopic surveys observe the combination of recession 
velocities due to the expansion of the Universe and the peculiar velocities of the galaxies 
\beq
z_{obv} = z_{true} +  \frac{v_{pec}}{c}.
\eeq 
The comoving positions derived from the angular positions and redshifts are then
in {\em redshift-space} and ``distorted'' compared to real-space comoving positions:
\beq
{\bf s} = {\bf x} +  \frac{{\bf v}_{pec} \cdot \hat{n}}{H_0}.
\eeq 
$\hat{n}$ is the unit vector along the line-of-sight. Thankfully, all hope is not 
lost. The peculiar velocities of galaxies should be directly related to the total 
matter distribution, since galaxies can be thought of as test particles in a 
gravitational field. 

\todo{Kaiser 87} derives an approximation for the distortion caused by the coherent 
infall of galaxies onto over dense regions in redshift space. This redshift-space 
distortion (RSD), often referred to as the Kaiser effect, causes overdense regions
to appear squashed along the line of sight in redshift space. Galaxies around an
overdense region closest to the observer (us) are moving towards the center of the 
overdense region, so they appear in redshift-space to be farther away. Galaxies 
on the other side of the overdense region are moving towards it and the observer, 
so they appear closer to us. 

More precisely, the relation between the overdensity field in redshift-space can be 
derived from the continuity equation and the distant observer approximation, 
\beq
\delta^{(s)}({\bf k}) = (1 + f \mu^2) \delta({\bf k}).
\eeq
$f$ here is the growth rate of structure from Eq.~\ref{eq:f_growth} and 
$\mu = {\bf k} \cdot \hat{n} / k$, cosine of the angle between $k$ and 
the line of sight. 

The Kaiser effect can be combined with the galaxy bias model from 
Section~\ref{sec:lss}, in order to express the galaxy overdensity field in 
redshift-space:
\beq
\delta_g^{(s)}({\bf k}) = (b + f \mu^2) \delta({\bf k}).
\eeq
The redshift-space powerspectrum of the galaxy overdensity field can then be
written as 
\beq
P_g^{(s)}(k, \mu) = (b + f \mu^2)^2 P(k).
\eeq
On large scales and with small overdensities, the effect of redshift-space 
distoritons is well described by the Kaiser effect. On small scales with large
overdensities things get a little more complicated. 

The random peculiar velocities of galaxies in gravitationally bound structures 
such as clusters cause their position in redshift-space along the line-of-sight 
to be smeared out to larger scales.  This effect can easily be identified by 
eye in galaxy redshift maps. The elongations of the galaxy positions along the 
line-of-sight look like fingers pointing towards the observer. Hence this 
redshift-space distorion is called the ``fingers-of-god''. Its impact on the 
powerspectrum, is typically quantified using an overall exponential factor such as 
\beq
P_g^{(s)}(k, \mu) \approx e^{-f^2 \sigma_v^2 \mu^2 k^2} (b + f \mu^2)^2 P(k).
\eeq
$\sigma_v$ is a paramter quantifying the strength of the effect and is usually
left as a free parameter. 

The relations that quantify the impact of RSDs reveal another means of measuring 
$f$. Consider the Legendre expansion of $P_g^{(s)}(k, \mu)$, 
\beq
P_g^{(s)}(k, \mu) = \sum\limits_{\ell=0, 2, 4 ...} \mathcal{L}_\ell(\mu) P_g^\ell(k). 
\eeq
Each of the powerspectrum ``multipoles'' of this expansion can be written as 
\beq
P_g^{\ell}(k) = \frac{2 \ell + 1}{2} \int\limits_{-1}^{1} {\rm d}\mu \; P_g^{(s)}(k, \mu) \mathcal{L}_\ell(\mu).
\eeq
The powerspectrum multipoles for $\ell= 0$ (monopole) and $2$ (quadrupole), neglecting 
the figers-of-god which does not significantly impact larger scales, are
\beqa
P_g^0 (k) = (b_1^2 + \frac{2}{3} f b_1 + \frac{1}{5}f^2) P(k) \\
P_g^2 (k) = (\frac{4}{3} f b_1 + \frac{4}{7} f^2) P(k). 
\eeqa
Their ratio 
\beq
\frac{P_g^2}{P_g^0} = \frac{\frac{4}{3} f b_1 + \frac{4}{7} f^2}{b_1^2 + \frac{2}{3} f b_1 + \frac{1}{5}f^2},
\eeq
illustrates how the distortions caused by RSDs allow us to extract information of 
$f$ through measurements of the redshift-space galaxy powerspectrum!  

%Beyond the general description and derivation of the redshift-space galaxy powerspectrum, the rest of galaxy clustering analysis in LSS studies follows the standard approach to Bayesian parameter inference. 
The ultimate goal of galaxy clustering analyses is to derive constraints on cosmological 
parameters and models from observed measurements of galaxly clustering -- the probability 
distribution of the cosmological parameters given observations. The standard approach
to deriving this {\em posterior} probability distribution is using Bayesian parameter
inference.  Based on Bayes theorem, the posterior is 
\beq
P({\bf \theta}| {\bf D}) = \frac{P({\bf D}|{\bf \theta}) P({\bf \theta})}{P({\bf D})}.
\eeq
${\bf D}$ and ${\bf \theta}$ refer to observations and cosmological parameters, respectively. 
$P({\bf D}|{\bf \theta})$, the probability of observing ${\bf D}$ given model 
parameters ${\bf \theta}$ is the {\em likelihood function}, $\mathcal{L}$. $P({\bf \theta})$ is 
the {\em prior} probability distribution. Lastly, $P({\bf D})$ is the ``evidence''. 
Since the evidence does not depend on ${\bf theta}$ it is just a normalization 
factor, so the equation above is more simply,  
\beq \label{eq:bayes} 
P({\bf \theta}| {\bf D}) \propto P({\bf D}|{\bf \theta}) P({\bf \theta}). 
\eeq

The likelihood, in galaxy clustering analyses, is {\em typically} assumed to have 
Gaussian function form and calculated as 
\beq
P({\bf D}|{\bf \theta}) = \mathcal{L} = \frac{1}{(2\pi)^{N/2} det{\bf C})^{1/2}}\; {\rm exp}\left( -\frac{1}{2} ({\bf D} - f({\bf \theta}))^T {\bf C}^{-1} ({\bf D} - f({\bf \theta}))\right).
\eeq
\todo{Talk about the difference between the model and observations. 
Then talk about the covariance matrix. Say what it is qualatatively.}

\todo{paragraph about how covariance matrices for LSS analysis are derived
using thousands of mock catalogs (hartlap 2007).}

Using the above likelihood, the posterior distribution of the parameters
of the model is derived. \todo{brief overview of the latest constraints from 
BOSS to summarize the state of the field now.} 

%The peculiar velocity component anisotropically ``distorts'' redshift-space galaxy clustering. Measuring this redshift-space distortion (RSD) allows us to infer the growth of structure, which we can then use to test GR and modified gravity scenarios. Galaxy clustering also provides a unique window to probe fundamental physics {\em i.e.} \mneut. Neutrinos suppress the growth of structure below their free-streaming scale and leave imprints on the LSS. From the imprints on galaxy clustering, we can place limits on \mneut. 
\section{Beyond Standard Galaxy Clustering Analysis}%Galaxy-Halo Connection}

\begin{enumerate}
{\item 
limits of the current framework
}
{\item 
mention bispectrum
}
{\item 
mention chapters and how they improve things
}
\end{enumerate}

\chapname s~\chapalt{fc} and~\chapalt{galenv} have both been refereed and
published in the astronomical literature.
\chapname s~\chapalt{abc} and~\chapalt{galhalo} have been submitted to the 
\emph{Monthly Notices of the Royal Astronomical Society} and \emph{The Astrophysical Journal},
respectively, and updated in response to the referees' comments.
All of these \chapname s were co-authored with collaborators but the majority
of the work and writing in each \chapname\ is mine.
Below, I describe my contributions to each \chapname:
\begin{enumerate}

{\item 
For \chap{fc}, I developed the idea for the project in collaboration with Roman
Scoccimarro and Michael Blanton. I implemented the project with contributions 
from Roman Scoccimarro. The project utilized simulation data from Jeremy Tinker
and Sergio Rodr\'{i}guez-Torres. I wrote the paper with additions from 
Roman Scoccimarro and edits by Michael Blanton. 
}

{\item 
For \chap{abc}, I developed the idea for the project in collaboration with 
Mohammadjavad Vakili, Andrew Hearin, and David Hogg. I implemented the project 
with Mohammadjavad Vakili and contributions from Andrew Hearin and Kilian Walsh.
The project utilized software written by Andrew Hearin and Duncan Campell. 
I wrote the paper together with Mohammadjavad Vakili with additions from
Andrew Hearin, David Hogg, and Kilian Walsh.
}

{\item 
For \chap{galenv}, I developed the idea for the project in collaboration with 
Michael Blanton. I implemented the project using catalogs constructed by 
John Moustakas from observations made by the PRIMUS collaboration (Alison Coil,
Richard Cool, Daniel Eisenstein, Ramin Skibb, Kenneth Wong, and Guangtun Zhu).
I wrote the paper with additions from Michael Blanton. 
}

{\item 
For \chap{galhalo}, I developed the idea for the project in collaboration with 
Jeremy Tinker. I implemented the project using simulation data from Andrew 
Wetzel. I wrote the paper with comments and edits by Jeremy Tinker and Andrew 
Wetzel. 
}
\end{enumerate}
